\documentclass[11pt,oneside,a4paper]{article}
\usepackage{graphicx}
\usepackage[utf8]{inputenc}
\usepackage[frenchb]{babel}
\usepackage{multirow}

\begin{document}

\part{Constante}
\begin{verbatim}
	VALUE_LIST = ("7","8","9","10","Valet","Dame","Roi","As") 
	COLOR_LIST = ("carreau", "pique", "trefle", "coeur")
	POINTS = (0,0,0,10,2,3,4,11)
	POINTS_ATOUTS = (0,0,14,10,20,3,4,11)
	VALID_DISTRIBUTION_SCHEMA = ( (3,3,2), (3,2,3), (2,3,3) )
\end{verbatim}

%------------------------------------------------------------------------------------------------
%------------------------------------------------------------------------------------------------
%------------------------------------------------------------------------------------------------

\part{Classes}

\section{Card}
Classe pour manipuler une carte.

\subsection{Attributs}
	\subsubsection{Value} Valeur de la carte représenté par un int dans l'intervale [0,7] mappé sur VALUE\_LIST
	\subsubsection{Color}
	Couleur de la carte représenté par un int dans l'intervale [0,3] mappé sur COLOR\_LIST

\subsection{Méthodes}
	
%------------------------------------------------------------------------------------------------
\section{Deck}
Classe représentant le jeu par une liste de 32 objet de type \textit{Card}

\subsection{Attributs}
	\subsubsection{cardlist} Liste des cartes dans le deck
	
\subsection{Méthodes}
	\subsubsection{deck.shuffle()} Mélange le jeux
	\subsubsection{deck.cut(mode="Auto",coupe=16,err=3)}
		Coupe le jeux suivant 3 méthodes
		\begin{description}
			\item[Auto :] le bot choisit
			\item[Flou :] le joueur choisit et le bot ajoute un marge d'erreurs
			\item[Fin  :] le joueur choisit précisement
		\end{description}
	\subsubsection{deck.distribute(schema=(3,2,3))}
		Distribue les cartes suivant le schéma spécifié en retournant 4 objet de type \textit{Hand} représentant les 4 mains des joueurs. \`A la fin de la distribution le deck est vide. Les schémas valides sont spécifié dans la constante VALID\_DISTRIBUTION\_SCHEMA
		
%------------------------------------------------------------------------------------------------
\section{Hand}
\subsection{Attributs}
	\subsubsection{cardlist} Liste des cartes dans une main
\subsection{Méthodes}
	\subsubsection{hand.play(index)} Retire la carte cardlist[index] de la main.
%------------------------------------------------------------------------------------------------
\section{}
\subsection{Attributs}

\subsection{Méthodes}

%------------------------------------------------------------------------------------------------
\section{}
\subsection{Attributs}

\subsection{Méthodes}

%------------------------------------------------------------------------------------------------
\section{}
\subsection{Attributs}

\subsection{Méthodes}
\end{document}